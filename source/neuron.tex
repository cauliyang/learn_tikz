\documentclass[tikz, border=0.2cm]{standalone}
\usepackage{tkz-graph}
\usepackage{amsmath,amssymb}
\usepackage{xcolor}
\usetikzlibrary{calc}
\usetikzlibrary{positioning}

\begin{document}


\newcommand{\inputnum}{3}
\newcommand{\hiddennum}{5}
\newcommand{\outputnum}{2}

\begin{tikzpicture}[
		mynode1/.style = { circle, minimum size = 6mm, fill=orange!30},
		mynode2/.style = { circle, minimum size = 6mm, fill=teal!50},
		mynode3/.style = { circle, minimum size = 6mm, fill=purple!50},
	]


	\foreach \i in {1,..., \inputnum}
		{
			\node[mynode1](Input-\i) at (0, -\i) {};
		}


	\foreach \i in {1, 2, ...,\hiddennum}{
			\node[mynode2,yshift=(\hiddennum - \inputnum)*5mm] (Hidden-\i) at (2.5, -\i){};
		}

	\foreach \i in {1, ...,\outputnum}{
			\node[mynode3,yshift=(\outputnum - \inputnum)*5mm] (Output-\i) at (5, -\i) {};
		}


	\foreach \i in {1, ..., \inputnum}{
			\foreach \j in {1, ..., \hiddennum}{
					\draw [->, shorten >=1pt] (Input-\i) -- (Hidden-\j);
				}
		}

	\foreach \i in {1, ..., \hiddennum}{
			\foreach \j in {1, ..., \outputnum}{
					\draw [->, shorten >=1pt] (Hidden-\i) -- (Output-\j);
				}
		}

	\foreach \i in {1, ..., \inputnum}{
			\draw [<-, shorten <=1pt] (Input-\i) -- ++(-1,0) node[left]{$x_{\i}$};
		}


	\foreach \i in {1, ...,\outputnum}{
			\draw [->, shorten <=1pt] (Output-\i) -- ++(1,0) node[right] {\(y_{\i}\)};
		}
\end{tikzpicture}

\end{document}





























































