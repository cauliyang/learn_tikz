% !TeX program = LuaLaTeX
\documentclass{article}

\usepackage{xcolor}
\usepackage{float}
% Load the LuaTeX packages
\usepackage{fontspec}
\usepackage{luatextra}
\usepackage{standalone}
\usepackage{amsmath,amssymb}
\usepackage{hvlogos}
\usepackage{pgfplots}
\usepgfplotslibrary{fillbetween}
\pgfplotsset{compat=newest}

\usepackage{tikz}
\usepackage{tkz-graph}
\usepackage{tikzpeople}

\usetikzlibrary{calc}
\usetikzlibrary{positioning}
\usetikzlibrary{arrows.meta}
\usetikzlibrary{shapes}
\usetikzlibrary{quotes}
\usetikzlibrary{graphs,graphdrawing}
\usetikzlibrary{trees}
\usetikzlibrary{shadings}
\usetikzlibrary{matrix}
\usetikzlibrary{mindmap}
\usetikzlibrary{intersections}
\usetikzlibrary{tikzlings}

\usegdlibrary{trees}

\usepackage{graphicx}
\graphicspath{{./source/}}

\usepackage[colorlinks=true]{hyperref}

% Set the main document font
\setmainfont{Latin Modern Roman}
\title{My Tikz Learning}
\author{Yangyang Li}
\date{\today}

\begin{document}
\maketitle
\clearpage
\tableofcontents

% <!-- begin content -->

\section{Case transformer}

\begin{figure}[H]
    \centering
    \includestandalone[width=\textwidth]{transformer}
    \caption{transformer \href{https:/github.com/cauliyang/learn_tikz/blob/main/source/source/transformer.tex}{code} }
    \label{fig:transformer}
\end{figure}

\section{Case hello world}

\begin{figure}[H]
    \centering
    \includestandalone[width=\textwidth]{hello_world}
    \caption{hello world \href{https:/github.com/cauliyang/learn_tikz/blob/main/source/source/hello_world.tex}{code} }
    \label{fig:hello world}
\end{figure}

\section{Case example1}

\begin{figure}[H]
    \centering
    \includestandalone[width=\textwidth]{example1}
    \caption{example1 \href{https:/github.com/cauliyang/learn_tikz/blob/main/source/source/example1.tex}{code} }
    \label{fig:example1}
\end{figure}

\section{Case image-overlay}

\begin{figure}[H]
    \centering
    \includestandalone[width=\textwidth]{image-overlay}
    \caption{image-overlay \href{https:/github.com/cauliyang/learn_tikz/blob/main/source/source/image-overlay.tex}{code} }
    \label{fig:image-overlay}
\end{figure}

\section{Case shape}

\begin{figure}[H]
    \centering
    \includestandalone[width=\textwidth]{shape}
    \caption{shape \href{https:/github.com/cauliyang/learn_tikz/blob/main/source/source/shape.tex}{code} }
    \label{fig:shape}
\end{figure}

\section{Case plot3d}

\begin{figure}[H]
    \centering
    \includestandalone[width=\textwidth]{plot3d}
    \caption{plot3d \href{https:/github.com/cauliyang/learn_tikz/blob/main/source/source/plot3d.tex}{code} }
    \label{fig:plot3d}
\end{figure}

\section{Case qual graph}

\begin{figure}[H]
    \centering
    \includestandalone[width=\textwidth]{qual_graph}
    \caption{qual graph \href{https:/github.com/cauliyang/learn_tikz/blob/main/source/source/qual_graph.tex}{code} }
    \label{fig:qual graph}
\end{figure}

\section{Case plot function}

\begin{figure}[H]
    \centering
    \includestandalone[width=\textwidth]{plot_function}
    \caption{plot function \href{https:/github.com/cauliyang/learn_tikz/blob/main/source/source/plot_function.tex}{code} }
    \label{fig:plot function}
\end{figure}

\section{Case tree2}

\begin{figure}[H]
    \centering
    \includestandalone[width=\textwidth]{tree2}
    \caption{tree2 \href{https:/github.com/cauliyang/learn_tikz/blob/main/source/source/tree2.tex}{code} }
    \label{fig:tree2}
\end{figure}

\section{Case curve}

\begin{figure}[H]
    \centering
    \includestandalone[width=\textwidth]{curve}
    \caption{curve \href{https:/github.com/cauliyang/learn_tikz/blob/main/source/source/curve.tex}{code} }
    \label{fig:curve}
\end{figure}

\section{Case plot func}

\begin{figure}[H]
    \centering
    \includestandalone[width=\textwidth]{plot_func}
    \caption{plot func \href{https:/github.com/cauliyang/learn_tikz/blob/main/source/source/plot_func.tex}{code} }
    \label{fig:plot func}
\end{figure}

\section{Case tree1}

\begin{figure}[H]
    \centering
    \includestandalone[width=\textwidth]{tree1}
    \caption{tree1 \href{https:/github.com/cauliyang/learn_tikz/blob/main/source/source/tree1.tex}{code} }
    \label{fig:tree1}
\end{figure}

\section{Case latex graphics tikz book5}

\begin{figure}[H]
    \centering
    \includestandalone[width=\textwidth]{latex_graphics_tikz_book5}
    \caption{latex graphics tikz book5 \href{https:/github.com/cauliyang/learn_tikz/blob/main/source/source/latex_graphics_tikz_book5.tex}{code} }
    \label{fig:latex graphics tikz book5}
\end{figure}

\section{Case latex graphics tikz book4}

\begin{figure}[H]
    \centering
    \includestandalone[width=\textwidth]{latex_graphics_tikz_book4}
    \caption{latex graphics tikz book4 \href{https:/github.com/cauliyang/learn_tikz/blob/main/source/source/latex_graphics_tikz_book4.tex}{code} }
    \label{fig:latex graphics tikz book4}
\end{figure}

\section{Case latex graphics tikz book1}

\begin{figure}[H]
    \centering
    \includestandalone[width=\textwidth]{latex_graphics_tikz_book1}
    \caption{latex graphics tikz book1 \href{https:/github.com/cauliyang/learn_tikz/blob/main/source/source/latex_graphics_tikz_book1.tex}{code} }
    \label{fig:latex graphics tikz book1}
\end{figure}

\section{Case colored diagram}

\begin{figure}[H]
    \centering
    \includestandalone[width=\textwidth]{colored_diagram}
    \caption{colored diagram \href{https:/github.com/cauliyang/learn_tikz/blob/main/source/source/colored_diagram.tex}{code} }
    \label{fig:colored diagram}
\end{figure}

\section{Case neuron}

\begin{figure}[H]
    \centering
    \includestandalone[width=\textwidth]{neuron}
    \caption{neuron \href{https:/github.com/cauliyang/learn_tikz/blob/main/source/source/neuron.tex}{code} }
    \label{fig:neuron}
\end{figure}

\section{Case latex graphics tikz book3}

\begin{figure}[H]
    \centering
    \includestandalone[width=\textwidth]{latex_graphics_tikz_book3}
    \caption{latex graphics tikz book3 \href{https:/github.com/cauliyang/learn_tikz/blob/main/source/source/latex_graphics_tikz_book3.tex}{code} }
    \label{fig:latex graphics tikz book3}
\end{figure}

\section{Case latex graphics tikz book2}

\begin{figure}[H]
    \centering
    \includestandalone[width=\textwidth]{latex_graphics_tikz_book2}
    \caption{latex graphics tikz book2 \href{https:/github.com/cauliyang/learn_tikz/blob/main/source/source/latex_graphics_tikz_book2.tex}{code} }
    \label{fig:latex graphics tikz book2}
\end{figure}

\section{Case diamond}

\begin{figure}[H]
    \centering
    \includestandalone[width=\textwidth]{diamond}
    \caption{diamond \href{https:/github.com/cauliyang/learn_tikz/blob/main/source/source/diamond.tex}{code} }
    \label{fig:diamond}
\end{figure}
    % <!-- end content -->
\end{document}
