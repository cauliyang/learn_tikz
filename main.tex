% !TeX program = LuaLaTeX

\documentclass{article}


\usepackage{xcolor}
\usepackage{float}
% Load the LuaTeX packages
\usepackage{fontspec}
\usepackage{luatextra}
\usepackage{standalone}
\usepackage{tkz-graph}
\usepackage{amsmath,amssymb}
\usetikzlibrary{calc}
\usetikzlibrary{positioning}
\usetikzlibrary{arrows.meta}
\usetikzlibrary{shapes}
\usetikzlibrary{graphs,graphdrawing}
\usegdlibrary {trees}

\usepackage[colorlinks=true]{hyperref}

% Set the main document font
\setmainfont{Latin Modern Roman}
\title{My Tikz Learning}
\author{Yangyang Li}


\begin{document}

\maketitle

\clearpage

\tableofcontents

% <!-- begin content -->

\section{Example transformer}

\begin{figure}[H]
	\centering
	\includestandalone[width=\textwidth]{transformer}
	\caption{transformer \href{https://github.com/cauliyang/learn_tikz/blob/main/transformer.tex}{code} }
	\label{fig:transformer}
\end{figure}

\section{Example hello world}

\begin{figure}[H]
	\centering
	\includestandalone[width=\textwidth]{hello_world}
	\caption{hello world \href{https://github.com/cauliyang/learn_tikz/blob/main/hello_world.tex}{code} }
	\label{fig:hello world}
\end{figure}

\section{Example qual graph}

\begin{figure}[H]
	\centering
	\includestandalone[width=\textwidth]{qual_graph}
	\caption{qual graph \href{https://github.com/cauliyang/learn_tikz/blob/main/qual_graph.tex}{code} }
	\label{fig:qual graph}
\end{figure}

\section{Example tree1}

\begin{figure}[H]
	\centering
	\includestandalone[width=\textwidth]{tree1}
	\caption{tree1 \href{https://github.com/cauliyang/learn_tikz/blob/main/tree1.tex}{code} }
	\label{fig:tree1}
\end{figure}

\section{Example colored diagram}

\begin{figure}[H]
	\centering
	\includestandalone[width=\textwidth]{colored_diagram}
	\caption{colored diagram \href{https://github.com/cauliyang/learn_tikz/blob/main/colored_diagram.tex}{code} }
	\label{fig:colored diagram}
\end{figure}
% <!-- end content -->


\end{document}


































































